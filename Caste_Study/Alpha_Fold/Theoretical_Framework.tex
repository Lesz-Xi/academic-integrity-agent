\documentclass[preprint,12pt]{elsarticle}

\usepackage{lineno,hyperref}
\usepackage{graphicx}
\usepackage{geometry}
\usepackage{booktabs}
\geometry{
 a4paper,
 total={170mm,257mm},
 left=20mm,
 top=20mm,
}
\modulolinenumbers[5]

\journal{Journal of Theoretical Virology}

\begin{document}

\begin{frontmatter}

\title{Proteolytic Immunoliposomes: A Theoretical Framework for Enzymatic Inactivation of HIV-1 via gp120 Cleavage}

\author{Rhine Lesther Tague}
\address{Department of Computer Science}

\begin{abstract}
\textbf{Background:} HIV-1 entry is mediated by the envelope glycoprotein gp120, whose conserved receptor-binding regions are susceptible to proteolytic cleavage by endosomal cysteine cathepsins. Immunoliposomes have been used to target HIV virions and infected cells, and Antibody-Directed Enzyme Prodrug Therapy (ADEPT) establishes principles for antibody-directed enzyme delivery and control.

\textbf{Objective:} To propose a theoretical design for proteolytic immunoliposomes that display anti-gp120 targeting moieties and carry cysteine proteases to catalytically cleave gp120 and inactivate virions.

\textbf{Approach:} We synthesize evidence that gp120 harbors conserved cathepsin cleavage sites at functionally critical regions, that immunoliposomes can target gp120/CD4/HLA-DR and deliver antiviral payloads \textit{in vitro} and \textit{in vivo}, and that ADEPT provides mechanistic guidance on enzyme targeting, activation, and clearance.

\textbf{Conclusion:} We outline a testable framework for protease-bearing, anti-gp120 immunoliposomes as a novel antiviral modality. \textbf{This is a theoretical proposal; experimental validation is pending.}
\end{abstract}

\begin{keyword}
HIV-1 \sep gp120 \sep Cysteine Protease \sep Immunoliposome \sep ADEPT \sep Viral Inactivation
\end{keyword}

\end{frontmatter}

\linenumbers

%% ============================================
%% INTRODUCTION
%% ============================================
\section{Introduction}

The HIV-1 envelope glycoprotein gp120 engages CD4 and coreceptors to initiate viral entry. Proteolysis of gp120 at receptor-binding and neutralizing-epitope regions can disrupt these functions and potentially abrogate infectivity.

Experimental studies show that cathepsins L and S (cysteine proteases) cleave gp120 at conserved, functionally important sites and reduce CD4 and neutralizing antibody binding, supporting the plausibility of enzymatic inactivation strategies targeting gp120 \cite{yu2008,steers2012}.

Immunoliposomes, decorated with anti-gp120 or anti-CD4 ligands, have achieved HIV neutralization and drug delivery \textit{in vitro} and reduced viremia \textit{in vivo} when targeted to lymphoid reservoirs \cite{wang2022,asmal2011}. ADEPT literature provides a mechanistic precedent for antibody-directed enzyme delivery and emphasizes safety features such as clearing antibodies and short-acting effectors \cite{bagshawe2006,francis2002}.

Collectively, these observations motivate a theoretical framework for proteolytic immunoliposomes to inactivate HIV-1 by gp120 cleavage.

%% ============================================
%% METHODS (THEORETICAL FRAMEWORK)
%% ============================================
\section{Methods (Theoretical Framework)}

\subsection{Design Concept}
Create PEGylated immunoliposomes displaying high-affinity anti-gp120 binders (e.g., CD4-binding-site VHHs or broadly neutralizing antibody Fabs) to dock on virion gp120, co-formulated with or bearing a membrane-tethered cysteine protease (e.g., cathepsin L/S engineered for neutral pH activity and external orientation).

The enzyme would cleave gp120 at conserved cathepsin-sensitive sites, diminishing receptor engagement. Design choices leverage evidence that gp120 contains conserved cathepsin L/S cleavage sites in C2, V3, and C4 domains and that cathepsin activity can disable antibody and CD4 interactions \cite{yu2008}.

\subsection{Targeting Module}
Use non-covalent or flexible linkers for antibody attachment to preserve neutralization potency and allow multivalent engagement, as suggested by differential performance of covalent versus chelating attachments in VHH-liposomes \cite{wang2022}. Alternative targeting to CD4 or HLA-DR may enhance reservoir access.

\subsection{Enzyme Control and Safety}
Apply ADEPT-inspired strategies:
\begin{itemize}
    \item Proenzyme formats activatable upon virion binding or endosomal uptake
    \item Inclusion of enzyme inhibitors or decoys for systemic control
    \item Potential clearing agents to remove circulating enzyme-bearing liposomes
\end{itemize}
These measures limit off-target proteolysis \cite{bagshawe2006,francis2002}.

%% ============================================
%% RESULTS (THEORETICAL ANALYSIS)
%% ============================================
\section{Results (Theoretical Analysis)}

\subsection{Mechanistic Plausibility}
Cathepsins L/S (cysteine proteases) have been shown to cleave gp120 at conserved sites that overlap receptor/neutralizing regions; digestion reduces binding to CD4 and neutralizing antibodies, implying loss of entry function upon targeted cleavage \cite{yu2008,steers2012}.

The presence of secreted cathepsins and their activity at or near neutral pH in certain contexts further supports feasibility if engineered appropriately for extracellular action.

\subsection{Targeted Delivery Feasibility}
Anti-gp120-decorated liposomes neutralize HIV and deliver antiretroviral cargo \textit{in vitro}, while immunoliposomes targeted to HLA-DR delivered a protein therapeutic \textit{in vivo} and reduced viremia in a nonhuman primate model \cite{wang2022,asmal2011}. This suggests that antibody-targeted liposomes can reach relevant compartments and exert antiviral effects.

\subsection{Anticipated Performance}
A protease payload offers catalytic action; once docked via anti-gp120 binders, localized proteolysis could inactivate multiple virions per particle, potentially overcoming low spike density and heterogeneity in epitope exposure.

ADEPT experience indicates the critical importance of rapid enzyme inactivation/clearance to avoid systemic toxicity and the need to mitigate immunogenicity of enzyme constructs \cite{francis2002}.

%% ============================================
%% FIGURE: EVIDENCE TABLE
%% ============================================
\begin{figure}[h!]
\centering
\includegraphics[width=\textwidth]{image-4.png}
\caption{Summary of retrieved evidence for gp120-targeted immunoliposomes, liposomal entry-inhibitor formulations, and cysteine (cathepsin) cleavage of gp120, with citations to the gathered sources showing where each claim is supported.}
\label{fig:evidence_table}
\end{figure}

%% ============================================
%% DISCUSSION
%% ============================================
\section{Discussion}

\subsection{Feasibility Analysis}
Evidence that gp120 is efficiently cleaved by cathepsins L/S and that such cleavage disrupts CD4/antibody binding supports the central mechanism. Immunoliposome studies show that antibody-decorated liposomes can bind gp120 and deliver antivirals, and that immunoliposome delivery can reduce viremia \textit{in vivo} when targeted to lymphoid cells.

ADEPT provides operational templates for enzyme targeting, activation control, and post-target clearance. Together, these lines of evidence support the theoretical feasibility of proteolytic immunoliposomes.

\subsection{Engineering Considerations}
Key variables include:
\begin{itemize}
    \item Protease selection/engineering for extracellular stability and specificity
    \item Orientation and density of targeting ligands
    \item Linker flexibility and attachment chemistry to preserve binding
    \item Formulation stability
\end{itemize}

The finding that non-covalent presentation preserved neutralization suggests flexible or releasable attachments could maximize gp120 engagement. HLA-DR or CD4 targeting may enhance access to reservoirs or infected cells \cite{wang2022}.

\subsection{Safety and Translational Risks}
ADEPT trials underscore risks of systemic toxicity and immunogenicity, emphasizing the need for:
\begin{itemize}
    \item Proenzyme designs
    \item Local activation
    \item Short half-life effectors
    \item Clearing agents
\end{itemize}

Manufacturing and regulatory considerations will require stringent control of enzyme activity and biodistribution \cite{francis2002}.

%% ============================================
%% LIMITATIONS
%% ============================================
\section{Limitations}

Direct evidence that protease-bearing, anti-gp120 immunoliposomes cleave virion-associated gp120 and abrogate infectivity is lacking. Cathepsin digestion studies primarily used soluble gp120 and immunochemical assays, not intact virions in physiological milieus.

\textit{In vivo} delivery of active proteases poses risks of off-target proteolysis; ADEPT-informed control strategies remain to be tested in the HIV setting.

\textbf{This is a theoretical proposal; experimental validation is pending.}

%% ============================================
%% CONCLUSION
%% ============================================
\section{Conclusion}

Proteolytic immunoliposomes that co-localize cysteine proteases with gp120 via anti-gp120 targeting represent a mechanistically grounded theoretical strategy to inactivate HIV-1.

Published data demonstrate:
\begin{enumerate}
    \item gp120 susceptibility to cathepsin cleavage at conserved, functional sites
    \item Feasibility of immunoliposome targeting and antiviral delivery
    \item ADEPT-derived principles for safe enzyme targeting
\end{enumerate}

Systematic experimental work is warranted to evaluate protease engineering, liposomal display and delivery, efficacy against cell-free and cell-associated virus, and safety controls inspired by ADEPT.

\textbf{This is a theoretical proposal; experimental validation is pending.}

%% ============================================
%% REFERENCES
%% ============================================

\begin{thebibliography}{10}

\bibitem{yu2008}
Yu Y, et al. Mapping conserved cathepsin L/S/D cleavage sites on HIV-1 gp120 and reduced CD4/neutralizing antibody binding upon cleavage. \textit{Journal of Virology}. 2008.

\bibitem{steers2012}
Steers NJ, et al. Env-A244 gp120 is readily cleaved by cathepsins L/S, generating immunogenic peptide arrays. \textit{Vaccine}. 2012.

\bibitem{wang2022}
Wang L, et al. Anti-gp120 VHH immunoliposomes neutralize HIV and deliver dapivirine \textit{in vitro}. \textit{Molecular Pharmaceutics}. 2022.

\bibitem{asmal2011}
Asmal M, et al. Reduction of viremia with anti-HLA-DR immunoliposome delivery of antithrombin III in nonhuman primates. \textit{PLoS ONE}. 2011.

\bibitem{bagshawe2006}
Bagshawe KD. Antibody-Directed Enzyme Prodrug Therapy (ADEPT): A Review. \textit{Clinical Cancer Research}. 2006.

\bibitem{francis2002}
Francis RJ, et al. A Phase I trial of antibody directed enzyme prodrug therapy (ADEPT) in patients with advanced colorectal carcinoma or other CEA producing tumours. \textit{British Journal of Cancer}. 2002.

\end{thebibliography}

\end{document}
